\documentclass[10pt]{article}
\usepackage{parskip}
\usepackage[utf8]{inputenc}
\usepackage[left=2.00cm, right=2.00cm, top=2.00cm, bottom=2.00cm]{geometry}
\usepackage[spanish]{babel}
\usepackage{graphicx,subfig}
\usepackage{fancyhdr}
\graphicspath{{Imagenes/}}
\usepackage{enumerate} 
\usepackage{multicol}
\usepackage{tabularx}
\usepackage{amssymb}
\usepackage{adjustbox}
\usepackage{amsmath}
\usepackage{cancel}
\begin{document}


\pagestyle{fancy}
\cfoot{}


%Cabeceras
\rhead{Oscilador Armonico.}
\lhead{}

%Portada
\begin{titlepage}
	\newgeometry{
		left=25mm,
		right=25mm,
		top=5mm,
		bottom=30mm,
		headheight = 0 mm
	}

	\begin{figure}[t]
		\subfloat{\includegraphics[width=0.15\textwidth]{Logo_IPN}}
		\hspace{0.6\textwidth}
		\subfloat{\includegraphics[width=0.22\textwidth]{LogoEsime}}
	\end{figure}

	\centering
	{\bfseries\Huge Instituto Politécnico Nacional. \par}
	\vspace{1cm}
	{\scshape\Large Ingeniería en Comunicaciones y Electrónica. \par}
	\vspace{0.3cm}
	{\scshape\Large Laboratorio de Ondas Mecanicas.  \par}
	\vspace{1cm}
	{\scshape\Huge Señor Doctor Slinky \par}
	\vspace{1cm}
	{\itshape\Large Oscilador Armonico. \par}
	{\Large 2CM13\par}
	\vfill
	{\Large Autores: \par}
	{\Large Hernández Huerta Jose Emilio. \par}
	{\Large Hernández Sanluis Danna Estefany.  \par}
	{\Large Garduño Bejarano Nataly. \par}
	{\Large Monijica Reyes Rogelio.\par}
	{\Large Morlan Juárez Bruno Tonatiuh.  \par}
	\vfill
	{\Large Sep 2023. \par}

\end{titlepage}

\tableofcontents
\newpage

\section{Resumen.}


\begin{multicols}{2}

\section{Objetivo.}



\section{Introducción.}



\section{Marco teórico.}

\section{Experimento 3.  Obtención de la aceleración de la gravedad de la localidad $(g)$}


\begin{center}
	\begin{adjustbox}{width=245pt}
		\begin{tabular}{|c|c|c|c|c|}
			\hline
			Resistencia & Colores & Tolerancia & Valor nominal (ohms) & Valor medido (ohms) \\
			\hline
			1 & Café, verde, rojo, dorado & $\pm$ 5\% & 1.5k & 1.46k \\
			\hline
			2 & Rojo, azul, rojo, dorado & $\pm$ 5\% & 2.6k & 2.18k \\
			\hline
			3 & Naranja, naranja, naranja, dorado & $\pm$ 5\% & 33k & 32.8k \\
			\hline
			4 & Amarillo, violeta, rojo, dorado & $\pm$ 5\% & 4.7k & 4.75k \\
			\hline
			5 & Azul, verde, rojo, dorado & $\pm$ 5\% & 6.5k & 6.66k \\
			\hline
		\end{tabular}
	\end{adjustbox}
\end{center}

\section{Conclusiones.}

\subsection*{Hernández Huerta Jose Emilio.}
Con base en lo experimentado con la aplicacion de un sistema masa resorte podemos calcular la aceleracion de la gravedad mediante calculos que involucran la posicion y su periodo, el cual tiene una peculiar coincidencia con la relacion entre sus masa y su periodo, pues son formas diferentes de allar el periodo entre con respecto a variables diferentes lo que me hace concluir que para despreciar la masa realmente trabajariamos con su peso $m*g$.
\subsection*{Morlan Juárez Bruno Tonatiuh.}
Durante este experimento observamos el fenómeno del Movimiento Armónico simple, aplicando fuerzas de diferente magnitud notando y comprobando su comportamiento con las fórmulas ya establecidas durante el experimento, los datos obtenidos han sido recolectados según lo aprendido en clases, solucionando algunas dudas durante el experimento se hizo aún más claro el tema, pero, el estudio y la dedicación harán que el curso vaya al mismo resultado. 
\subsection*{.}

\subsection*{Nataly Bejarano Garduño.}
En esta practica, se observa el Fenomeno del Movimiento Armonico Simple comprobando como tomar un periodo de onda y hacer una relción mediante graficas para observar esta relación de masa resorte, y apreciar la Ley Houk sobre Señor Doctor Slinky. 


\subsection*{Uriel Grimaldi Díaz.}
\begin{thebibliography}{0}
	\bibitem{citekey}[Bragado, I. M. (2003). Física General.]
	\bibitem{citekey}[Benchimol, D. (c. 2020). Electrónica práctica. USERSHOP.]
	\bibitem{citekey}[Peaktech. (2016). Manual de Usuario Peaktech 2005.]
		
\end{thebibliography}

\end{multicols}

\end{document}

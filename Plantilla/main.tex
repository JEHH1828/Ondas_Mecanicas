\documentclass[10pt]{article}
\usepackage{parskip}
\usepackage[utf8]{inputenc}
\usepackage[left=2.00cm, right=2.00cm, top=2.00cm, bottom=2.00cm]{geometry}
\usepackage[spanish]{babel}
\usepackage{graphicx,subfig}
\usepackage{fancyhdr}
\graphicspath{{Imagenes/}}
\usepackage{enumerate} 
\usepackage{multicol}
\usepackage{tabularx}
\usepackage{amssymb}
\usepackage{adjustbox}
\usepackage{amsmath}
\usepackage{cancel}
\begin{document}


\pagestyle{fancy}
\cfoot{}


%Cabeceras
\rhead{Ley de Ohm.}
\lhead{}

%Portada
\begin{titlepage}
	\newgeometry{
		left=25mm,
		right=25mm,
		top=5mm,
		bottom=30mm,
		headheight = 0 mm
	}

	\begin{figure}[t]
		\subfloat{\includegraphics[width=0.15\textwidth]{Logo_IPN}}
		\hspace{0.6\textwidth}
		\subfloat{\includegraphics[width=0.22\textwidth]{LogoEsime}}
	\end{figure}

	\centering
	{\bfseries\Huge Instituto Politécnico Nacional. \par}
	\vspace{1cm}
	{\scshape\Large Ingeniería en Comunicaciones y Electrónica. \par}
	\vspace{0.3cm}
	{\scshape\Large Laboratorio de Electricidad y Magnetismo.  \par}
	\vspace{1cm}
	{\scshape\Huge Victoria la Reina Insaciable \par}
	\vspace{1cm}
	{\itshape\Large Ley de Ohm. \par}
	{\Large 2CM13\par}
	\vfill
	{\Large Autores: \par}
	{\Large Daniela Elizabeth Pérez Vargas. \par}
	{\Large Jesús Martinez Amac. \par}
	{\Large José Emilio Hernández Huerta. \par}
	{\Large Nataly Bejarano Garduño.\par}
	{\Large Uriel Grimaldi Díaz.  \par}
	\vfill
	{\Large Junio 2023. \par}

\end{titlepage}

\tableofcontents
\newpage

\section{Resumen.}


\begin{multicols}{2}

\section{Objetivo.}



\section{Introducción.}



\section{Marco teórico.}

\subsection{El multímetro.}


\subsection{Antes de usar un multímetro.}


\subsection{Como utilizar el multímetro.}

\subsubsection{Diferencia de potencial eléctrico.}

\subsubsection{Corriente eléctrica.}


\subsubsection{Resistencia.}


\subsection{Ley de Ohm.}


\subsection{Código de colores en los resistores.}

\section{Descripción de materiales.}


\section{Desarrollo experimental.}

\subsection{Reconocimiento del multímetro.}


\subsection{Mediciones de resistencia(óhmetro).}


\subsection{Mediciones de continuidad.}

\subsection{Mediciones de diferencia de potencial eléctrico(vóltmetro).}

\subsection{Mediciones de diferencia de potencial eléctrico de corriente alterna(vóltmetro).}


\subsection{Mediciones de intensidad de corriente continua(Amperímetro.)}

\section{Discusión de materiales.}

\subsection{Diferencias entre multímetro digital y analógico}

\subsection{¿Pilas o acumuladores?}


\subsection{Protoboard}

\subsection{Variación en el valor nominal de los resistores.}

\section{Análisis y resultados.}

\subsection{Reconocimiento del multímetro.}
 

\subsubsection{Multímetro Digital Peaktech 2005.}


\subsection{Mediciones de resistencia(óhmetro).}

\begin{center}
	\begin{adjustbox}{width=245pt}
		\begin{tabular}{|c|c|c|c|c|}
			\hline
			Resistencia & Colores & Tolerancia & Valor nominal (ohms) & Valor medido (ohms) \\
			\hline
			1 & Café, verde, rojo, dorado & $\pm$ 5\% & 1.5k & 1.46k \\
			\hline
			2 & Rojo, azul, rojo, dorado & $\pm$ 5\% & 2.6k & 2.18k \\
			\hline
			3 & Naranja, naranja, naranja, dorado & $\pm$ 5\% & 33k & 32.8k \\
			\hline
			4 & Amarillo, violeta, rojo, dorado & $\pm$ 5\% & 4.7k & 4.75k \\
			\hline
			5 & Azul, verde, rojo, dorado & $\pm$ 5\% & 6.5k & 6.66k \\
			\hline
		\end{tabular}
	\end{adjustbox}
\end{center}

\subsection{Mediciones de continuidad.}


\subsection{Mediciones de diferencia de potencial eléctrico(vóltmetro).}


\subsection{Mediciones de diferencia de potencial eléctrico de corriente alterna(vóltmetro).}


\subsection{Mediciones de intensidad de corriente directa(amperímetro).}


\section{Conclusiones.}

\subsection*{José Emilio Hernández Huerta.}

\subsection*{Daniela Elizabeth Pérez Vargas.}

\subsection*{Jesús Martinez Amac.}

\subsection*{Nataly Bejarano Garduño.}


\subsection*{Uriel Grimaldi Díaz.}
\begin{thebibliography}{0}
	\bibitem{citekey}[Bragado, I. M. (2003). Física General.]
	\bibitem{citekey}[Benchimol, D. (c. 2020). Electrónica práctica. USERSHOP.]
	\bibitem{citekey}[Peaktech. (2016). Manual de Usuario Peaktech 2005.]
		
\end{thebibliography}

\end{multicols}

\end{document}
